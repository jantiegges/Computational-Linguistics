\documentclass[10pt]{article}

\usepackage[margin=1in]{geometry}
\usepackage{amsmath,amsthm,amssymb, graphicx, multicol, array, parskip}
\usepackage[T1]{fontenc}
\usepackage{textcomp}
\usepackage{url}
% use straight quotes in texttt environment, make 0 different from O
\usepackage[zerostyle=b,straightquotes,scaled=.93]{newtxtt}

\usepackage{listings}
\lstset{
 basicstyle=\ttfamily,
 columns=fullflexible,
 upquote,
 keepspaces,
 literate={*}{{\char42}}1
 {-}{{\char45}}1
}
\usepackage[short labels]{enumitem}

\usepackage{soul}  % for strike through (\st{})
\usepackage[dvipsnames,usenames,table]{xcolor} % for colors
% \usepackage[htt]{hyphenat} % to break lines in texttt

\usepackage[framemethod=TikZ]{mdframed}
\usepackage{mdframed}
\global\mdfdefinestyle{simple}{linewidth=1pt,skipabove=.5em}
\newenvironment{AnswerBox}{\begin{mdframed}[style=simple]}{\end{mdframed}}
\newcommand\defeq{\mathrel{\overset{\makebox[0pt]{\mbox{\normalfont\tiny\sffamily def}}}{=}}}
\usepackage[colorlinks]{hyperref} % \href{http://link.com}{link text}
\hypersetup{linkcolor=NavyBlue,citecolor=NavyBlue,filecolor=NavyBlue,urlcolor=NavyBlue}
\usepackage{dsfont}
\newcommand{\required}[1]{{\color{blue}{#1}}}
\newcommand{\email}[1]{\href{mailto:#1}{\texttt{#1}}}
\newcommand{\PSnum}{1}

\author{
  \textbf{Name}:       %Put your name here.
, \textbf{McGill ID}:  %Put your McGill ID here. 
\\ \textit{Collaborators}:  %Put the names of up to 2 people if you completed the assignment as a group.
}

\begin{document}

\title{LING/COMP 445, LING 645\\Problem Set \PSnum} \date{Due before 4:35 PM on
Wednesday, September 13, 2023} 
\maketitle 

Please enter your name and McGill ID
above. There are several types of questions below.
\begin{itemize}
  \item For questions involving answers in English or mathematics or a
    combination of the two, put your answers to the question in the answer box
    provided, like in the example below.

    This .pdf document you are reading was compiled from a .tex document with
    \LaTeX. You must use \LaTeX{} for this problem set, and future problem sets
    in this course.\footnote{To compile a file \texttt{file.tex} to
      \texttt{file.pdf}, you can install \LaTeX{}, and use the command
      \texttt{pdflatex file.tex} at the command line, or make use of an online
      service such as \url{https://overleaf.com}. You can find more information
    about \LaTeX{} here \url{https://www.latex-project.org/}.}

  \item For programming questions, please put your answers into a file called
    \texttt{ps\PSnum-lastname-firstname.clj}. Be careful to follow the
    instructions exactly and be sure that all of your function definitions use
    the precise names, number of inputs and input types, and output types as
    requested in each question.

    For the code portion of the assignment, \textbf{it is crucial to submit a
    standalone file that runs}. Before you submit
    \texttt{ps\PSnum-lastname-firstname.clj}, make sure that your code executes
    correctly without any errors when run at the command line by typing
    \texttt{clojure ps\PSnum-lastname-firtname.clj} at a terminal prompt. We
    cannot grade any code that does not run correctly as a standalone file, and
    if the preceding command produces an error, the code portion of the
    assignment will receive a $0$.

    To do the computational problems, we recommend that you install Clojure on
    your local machine and write and debug the answers to each problem in a
    local copy of \texttt{ps\PSnum-lastname-firstname.clj}. You can find
    information about installing and using Clojure here
    \url{https://clojure.org/}.
\end{itemize}

Once you have entered your answers, please compile your copy of this \LaTeX{}
file into a PDF and submit
\begin{enumerate}[(i),noitemsep]
  \item the compiled PDF renamed to \texttt{ps\PSnum-lastname-firstname.pdf}
  \item the raw \LaTeX{} file renamed to
    \texttt{ps\PSnum-lastname-firstname.tex} and
  \item your \texttt{ps\PSnum-lastname-firstname.clj}
\end{enumerate}
to the Problem Set \PSnum{} folder under `Assignments' on MyCourses.




\noindent\hrulefill%-------------Example-Problem----

\paragraph{Example Problem:} This is an example question using some fake math
like this $L=\sum_0^{\infty} \mathcal{G} \delta_x$.

\textbf{Note:} If you're new to \LaTeX{}, find the code corresponding to this
text in the .tex document to see an example of using \LaTeX{} to typeset math
(you'll need this for later problem sets).


\paragraph{Example Answer:} Put your answer in the box provided, like this:
\begin{AnswerBox}
  Example answer is $L=\sum_0^{\infty} \mathcal{G} \delta_x$.
\end{AnswerBox}


\clearpage\pagebreak

\noindent\hrulefill%--------------------------------


\paragraph{Problem 1:}
Write an expression which defines a variable \required{\texttt{year}} with the
integer value \texttt{2023}.

\paragraph{Answer 1:} Please put your answer in
\texttt{ps\PSnum-lastname-firstname.clj}.

\noindent\hrulefill%--------------------------------

\paragraph{Problem 2:} Consider the following:
\begin{lstlisting}
  (= 4 (+ 1 2))
\end{lstlisting}
This is:
\begin{enumerate}[nosep, label=\Alph*.]
  \item an expression
  \item a list
  \item both
  \item neither
\end{enumerate}

\paragraph{Answer 2:} Please put your answer in the box below.

\begin{AnswerBox}%% Do not delete %%%%%%%

    %%%%%%%%%%%%%%%%%%%%%%%
    %% Your answer (A, B, C, or D) here. You can delete this comment.
    %%%%%%%%%%%%%%%%%%%%%%%

\end{AnswerBox}%% Do not delete %%%%%%%

\hrulefill%--------------------------------

\paragraph{Problem 3:} Which of the following evaluates to a value (A, B, both,
or neither)?

\begin{enumerate}[nosep, label=\Alph*.]
  \item \texttt{'(2 2 2)}
  \item \texttt{(2 2 2)}
\end{enumerate}

\paragraph{Answer 3:} Please put your answer in the box below.

\begin{AnswerBox}%% Do not delete %%%%%%%

  %%%%%%%%%%%%%%%%%%%%%%%
  %% Your answer (A, B, both, or neither) here. You can delete this comment.
  %%%%%%%%%%%%%%%%%%%%%%%

\end{AnswerBox}%% Do not delete %%%%%%%

\hrulefill%--------------------------------

\paragraph{Problem 4:} Consider the following expression:
\begin{lstlisting}
  (= "4" (+ 1 3))
\end{lstlisting}

This expression contains:
\begin{enumerate}[nosep, label=\Alph*.]
  \item a string
  \item quoted material
  \item both
  \item neither
\end{enumerate}

\paragraph{Answer 4:} Please put your answer in the box below.

\begin{AnswerBox}%% Do not delete %%%%%%%

  %%%%%%%%%%%%%%%%%%%%%%%
  %% Your answer (A, B, C, or D) here. You can delete this comment.
  %%%%%%%%%%%%%%%%%%%%%%%

\end{AnswerBox}%% Do not delete %%%%%%%

\hrulefill%--------------------------------

\paragraph{Problem 5:} Write a function \required{\texttt{add-up}} that takes
two numbers returns their sum.

\paragraph{Answer 5:} Please put your answer in
\texttt{ps\PSnum-lastname-firstname.clj}.

\noindent\hrulefill%--------------------------------

\paragraph{Problem 6:} Write a function \required{\texttt{is-it-four?}} that
returns \texttt{true} when given the number \texttt{4}, and returns
\texttt{false} otherwise.

\textbf{Note}: Don't forget the question mark in the function name! This is a
convention in Clojure\footnote{See
\url{https://guide.clojure.style/\#naming-predicates}.} for the names of
predicate functions (functions that return a boolean value---\texttt{true} or
\texttt{false}). Also, an incorrectly named function won't be seen by the grader
script, so even if your function behaves correctly, you won't get credit if it
has the wrong name!

\paragraph{Answer 6:} Please put your answer in
\texttt{ps\PSnum-lastname-firstname.clj}.

\noindent\hrulefill%--------------------------------

\paragraph{Problem 7:} Fill in the blank, so the following expression evaluates
to \texttt{true}:
\begin{lstlisting}
  (= (quote ___) 'platypus)
\end{lstlisting}

\paragraph{Answer 7:} Please put your answer in
\texttt{ps\PSnum-lastname-firstname.clj}.

\noindent\hrulefill%--------------------------------

\paragraph{Problem 8:} Define a function \required{\texttt{func}} and an
expression \required{\texttt{expr}} such that the following evaluates to
\texttt{true}.
\begin{lstlisting}
  (= 3 (apply func expr))
\end{lstlisting}

\paragraph{Hint:} be sure you understand what kinds of arguments \texttt{apply}
expects.

\paragraph{Answer 8:} Please put your answer in
\texttt{ps\PSnum-lastname-firstname.clj}.

\noindent\hrulefill%--------------------------------

\paragraph{Problem 9:} The built-in function \texttt{type} is useful for
checking what kind of object an expression evaluates to.\footnote{Though note
  that the types are different in Clojure (in the command line) versus
  ClojureScript (in the textbook).  For instance, they disagree about whether
  integer numbers (like $4$) and  floating point numbers (like $4.0$) are
  different types. \textbf{For the purposes of this problem set, don't worry
about this}.} Write a function \required{\texttt{both-same-type?}} that takes
two arguments, and returns \texttt{true} when they both have the same type,
and \texttt{false} otherwise.

\paragraph{Answer 9:} Please put your answer in
\texttt{ps\PSnum-lastname-firstname.clj}.

\noindent\hrulefill%--------------------------------

\paragraph{Problem 10:} Write a function \required{\texttt{list-longer-than?}}
which takes two arguments: an integer \texttt{n}, and a list \texttt{lst} and
returns \texttt{true} if \texttt{lst} has more than \texttt{n} elements, and
\texttt{false} otherwise. For example, \texttt{(list-longer-than? 3 '(1 2 3))}
should return false, and \texttt{(list-longer-than? 2 '(1 2 3))} should return
true.

\paragraph*{Hint:} you may find built-in clojure function \texttt{count} useful.

\paragraph{Answer 10:} Please put your answer in
\texttt{ps\PSnum-lastname-firstname.clj}.

\noindent\hrulefill%--------------------------------

\paragraph{Problem 11:} In linear algebra, if $\mathbf x, \mathbf y$ are two
vectors each with $n$ components, their dot product is $\mathbf x\cdot \mathbf y
= \sum_{i=1}^n x_i y_i$. Write a function \required{\texttt{dot-product}} that
takes two lists of numbers as arguments, and returns the dot product. So for
example, if the list \texttt{x} is \texttt{'(0 2 4)} and the list \texttt{y} is
\texttt{'(1 3 5)}, the expression \texttt{(dot-product x y)} should return
\texttt{26}, because $0\cdot1 + 2\cdot3 + 4\cdot5 = 26$.

You may assume that the two input lists are of equal length and contain only
numbers as elements.

\paragraph*{Hint:} you may find built-in clojure functions \texttt{apply} and
\texttt{map} useful.

\paragraph{Answer 11:} Please put your answer in
\texttt{ps\PSnum-lastname-firstname.clj}.

\noindent\hrulefill%--------------------------------

\paragraph{Problem 12:} In Clojure (like other functional programming languages)
functions and variables are treated identically. This means a function may
easily take another function as an argument, and/or return a
function.\footnote{Functions which take functions as arguments or return
  functions are called `higher order functions'. Built-in functions \texttt{map}
and \texttt{apply} are higher order functions.} Write a function
\required{\texttt{swap-arg-order}} which takes a \emph{function} (of two
arguments) as an argument returns \emph{another function} that does the same
thing, but expects its two arguments in the opposite order.

That is, for example
\begin{itemize}[nosep]
  \item given the division function \texttt{/} which divides the first argument
    by the second (so \texttt{(/ 3 6)} returns the number \texttt{1/2}), the
    following expression should evaluate to \texttt{2}
    \begin{lstlisting}
  ((swap-arg-order /) 3 6)
    \end{lstlisting}
  \item given the function \texttt{list-longer-than?} from above, the following
    expression should evaluate to \texttt{true}
    \begin{lstlisting}
  ((swap-arg-order list-longer-than?) '(1 2 3) 2)
    \end{lstlisting}
\end{itemize}

\paragraph{Answer 12:} Please put your answer in
\texttt{ps\PSnum-lastname-firstname.clj}.

\noindent\hrulefill%--------------------------------

\paragraph{Problem 13:} Define a higher order function \required{\texttt{g}} so
the following expression evaluates to \texttt{true}:
\begin{lstlisting}
  (= 100 (g (fn [n] (* n n))))
\end{lstlisting}

\paragraph{Answer 13:} Please put your answer in
\texttt{ps\PSnum-lastname-firstname.clj}.

\noindent\hrulefill%--------------------------------

\end{document}
