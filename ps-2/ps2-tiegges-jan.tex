\documentclass[10pt]{article}

\usepackage[margin=1in]{geometry}
\usepackage{amsmath,amsthm,amssymb, graphicx, multicol, array, parskip}
\usepackage[T1]{fontenc}
\usepackage{textcomp}
\usepackage{url}
% use straight quotes in texttt environment, make 0 different from O
\usepackage[zerostyle=b,straightquotes,scaled=.93]{newtxtt}

\usepackage{listings}
\lstset{
 basicstyle=\ttfamily,
 columns=fullflexible,
 upquote,
 keepspaces,
 literate={*}{{\char42}}1
 {-}{{\char45}}1
}
\usepackage[short labels]{enumitem}

\usepackage{soul}  % for strike through (\st{})
\usepackage[dvipsnames,usenames,table]{xcolor} % for colors
% \usepackage[htt]{hyphenat} % to break lines in texttt

\usepackage[framemethod=TikZ]{mdframed}
\usepackage{mdframed}
\global\mdfdefinestyle{simple}{linewidth=1pt,skipabove=.5em}
\newenvironment{AnswerBox}{\begin{mdframed}[style=simple]}{\end{mdframed}}
\newcommand\defeq{\mathrel{\overset{\makebox[0pt]{\mbox{\normalfont\tiny\sffamily def}}}{=}}}
\usepackage[colorlinks]{hyperref} % \href{http://link.com}{link text}
\hypersetup{linkcolor=NavyBlue,citecolor=NavyBlue,filecolor=NavyBlue,urlcolor=NavyBlue}
\usepackage{dsfont}
\newcommand{\required}[1]{{\color{blue}{#1}}}
\newcommand{\email}[1]{\href{mailto:#1}{\texttt{#1}}}
\newcommand{\PSnum}{2}

\author{
  \textbf{Name}:       %Put your name here.
, \textbf{McGill ID}:  %Put your McGill ID here.
\\ \textit{Collaborators}:  %Put the names of up to 2 people if you completed the assignment as a group.
}

\begin{document}

\title{LING/COMP 445, LING 645\\Problem Set \PSnum}
\date{Due before 4:35 PM on Wednesday, September 27, 2023}
\maketitle

Please enter your name and McGill ID above.  There are several types of questions
below.

\begin{itemize}
  \item For questions involving answers in English or mathematics or a
    combination of the two, put your answers to the question in an answer box
    like in the example below.
  \item For programming questions, please put your answers into a file called
    \texttt{ps\PSnum-lastname-firstname.clj}. Be careful to follow the
    instructions exactly and be sure that all of your function definitions use
    the precise names, number of inputs and input types, and output types as
    requested in each question.

    For the code portion of the assignment, \textbf{it is crucial to submit a
    standalone file that runs}. Before you submit
    \texttt{ps\PSnum-lastname-firstname.clj},  make sure that your code executes
    correctly without any errors  when run at the command line by typing
    \texttt{clojure ps\PSnum-lastname-firtname.clj} at a terminal prompt. We
    cannot grade any code that does not run correctly as a standalone file, and
    if the preceding command produces an error, the code portion of the
    assignment will receive a $0$.

    To do the computational problems, we recommend that you install Clojure on
    your local machine and write and debug the answers to each problem in a
    local copy of \texttt{ps\PSnum-lastname-firstname.clj}. You can find
    information about installing and using Clojure here
    \url{https://clojure.org/}.

    {\color{red}{Note} there is a built-in function called \texttt{reverse}.
      \textbf{Do not use the built-in function \texttt{reverse} in this problem
      set!} We remove \texttt{reverse} from the namespace when we grade, so
    using it anywhere will lead to an error and a result in a $0$.}
\end{itemize}

Once you have entered your answers, please compile your copy of this \LaTeX{}
file\footnote{To compile a file \texttt{file.tex} to \texttt{file.pdf}, you can
  use the command \texttt{pdflatex file.tex} at the command line, or make use of
  an online service such as \url{https://overleaf.com}. You can find more
information about \LaTeX{} here \url{https://www.latex-project.org/}.} into a
PDF and submit
\begin{enumerate}[(i),noitemsep]
  \item the compiled PDF renamed to \texttt{ps\PSnum-lastname-firstname.pdf}
  \item the raw \LaTeX{} file renamed to
    \texttt{ps\PSnum-lastname-firstname.tex} and
  \item your \texttt{ps\PSnum-lastname-firstname.clj}
\end{enumerate}
to the Problem Set \PSnum{} folder under `Assignments' on MyCourses.

\noindent\hrulefill %--------------------------------

\paragraph{Example Problem:} This is an example question using some fake math
like this $L=\sum_0^{\infty} \mathcal{G} \delta_x$.

\paragraph{Example Answer:} Put your answer in the box provided, like this:
\begin{AnswerBox}
  Example answer is $L=\sum_0^{\infty} \mathcal{G} \delta_x$.
\end{AnswerBox}


\pagebreak%-------------------------------------------
\section*{MAIN PROBLEM SET}
\paragraph{Problem 1:}
Write a single-argument function called \required{\texttt{absval}} that, when
passed a number,  computes its absolute value.  It should do this by finding the
square root of the square of the argument.  (Note: you should use the
\texttt{Math/sqrt} function built in to Clojure (from Java), which returns the
square root of a number.)

\paragraph{Answer 1:} Please put your answer in
\texttt{ps\PSnum-lastname-firstname.clj}.

\noindent\hrulefill %--------------------------------

\paragraph{Problem 2:} In both of the following definitions, there are one or
more errors of some kind. In each case, explain what's wrong and why, and fix
it:

\begin{lstlisting}
  (defn take-square
    (* x x))
\end{lstlisting}

\begin{lstlisting}
  (defn sum-of-squares [(take-square x) (take-square y)]
    (+ (take-square x) (take-square y)))
\end{lstlisting}

\paragraph{Answer 2:} Please put the fixed functions in
\texttt{ps\PSnum-lastname-firstname.clj} and \textbf{describe what is wrong and
why in the box below}.

\begin{AnswerBox}%% Do not delete %%%%%%%

  %%%%%%%%%%%%%%%%%%%%%%%
  %% YOUR ANSWER HERE. %% (you may delete this comment)
  %%%%%%%%%%%%%%%%%%%%%%%

\end{AnswerBox}%% Do not delete %%%%%%%

\noindent\hrulefill %--------------------------------

\paragraph{Problem 3:}
The expression \texttt{(+ 11 2)}  evaluates to \texttt{13}. Write four
other different Clojure expressions which also evaluate to the number
\texttt{13} (either the integer \texttt{13} or the float \texttt{13.0}). Using
\texttt{def}, assign these expressions to the symbols
\required{\texttt{exp-13-1}},
\required{\texttt{exp-13-2}},
\required{\texttt{exp-13-3}}, and
\required{\texttt{exp-13-4}}.

In each def statement, be sure to quote the expression (as below for our
example), so it is not evaluated before being assigned to the symbol.

\begin{lstlisting}
  (def exp-13-0 '(+ 11 2))
\end{lstlisting}

\paragraph{Answer 3:} Please put your answer in
\texttt{ps\PSnum-lastname-firstname.clj}.

\noindent\hrulefill %--------------------------------

\paragraph{Problem 4:}
Define a function called \required{\texttt{third}}, that selects the third
element of a list. For example, given the list \texttt{'(4 5 6)} as its
argument, \texttt{third} should return the number \texttt{6}.

\paragraph{Answer 4:} Please put your answer in
\texttt{ps\PSnum-lastname-firstname.clj}.

\noindent\hrulefill %--------------------------------

\paragraph{Problem 5:} Define a function called \required{\texttt{compose}},
that takes two one-place functions \texttt{f} and \texttt{g} as arguments. It
should return a new function, the composition of its input functions, which
computes \texttt{f} of \texttt{g} of \texttt{x} when passed the argument
\texttt{x}. For example, the function \texttt{Math/sqrt} (built in to Clojure
from Java) takes the square root of a number, and the function \texttt{Math/abs}
(likewise)  takes the absolute value of a number. If we use \texttt{defn} to
define functions \texttt{sqrt} and \texttt{abs} as
\begin{lstlisting}
  (defn sqrt [x] (Math/sqrt x))
  (defn abs [x] (Math/abs x))
\end{lstlisting}
then \texttt{((compose sqrt abs) -36)} should return \texttt{6}, since the
square root of the absolute value of \texttt{-36} equals \texttt{6}.

\paragraph{Answer 5:} Please put your answer in \texttt{ps\PSnum-lastname-firstname.clj}.

\noindent\hrulefill %--------------------------------

\paragraph{Problem 6:}
Define a function called \required{\texttt{first-two}} that takes a list as its
sole argument, and returns a two-element list containing the first two elements
of the argument. For example, given the list \texttt{'(4 5 6)},
\texttt{first-two} should return the list \texttt{'(4 5)}.

You may assume that the list passed in has at least two elements.

\paragraph{Answer 6:} Please put your answer in
\texttt{ps\PSnum-lastname-firstname.clj}.

\noindent\hrulefill %--------------------------------

\paragraph{Problem 7:}
Define a function called \required{\texttt{remove-second}} that takes a list,
and returns a new list that is the same as the input list, but with  the second
value removed. For example, given \texttt{'(3 1 4)}, \texttt{remove-second}
should return the list \texttt{'(3 4)}.

\paragraph{Answer 7:} Please put your answer in
\texttt{ps\PSnum-lastname-firstname.clj}.

\noindent\hrulefill %--------------------------------

\paragraph{Problem 8:}
Define a function called \required{\texttt{add-to-end}} that takes in two
arguments: a list \texttt{lst} and a value \texttt{x}. It should return a new
list which is the same as \texttt{lst}, except that it has \texttt{x} appended
as its final element. For example, \texttt{(add-to-end (list 5 6 4) 0)} should
return the list \texttt{'(5 6 4 0)}.

\paragraph{Answer 8:} Please put your answer in
\texttt{ps\PSnum-lastname-firstname.clj}.

\noindent\hrulefill %--------------------------------

\paragraph{Problem 9:}
Define a function called \required{\texttt{reverse-list}}, that takes in a list,
and returns the reverse of the list. For example, if it takes in the list
\texttt{'(a b c)}, it will output the list \texttt{'(c b a)}.

Do not use the built-in function \texttt{reverse} (in this problem, nor anywhere
in this problem set).

\paragraph{Answer 9:} Please put your answer in
\texttt{ps\PSnum-lastname-firstname.clj}.

\noindent\hrulefill %--------------------------------

\paragraph{Problem 10:}
Define a function called \required{\texttt{count-to-1}}, that takes a positive
integer \texttt{n}, and returns a list of the integers counting down from
\texttt{n} to \texttt{1}. For example, given input \texttt{3}, it will return
the list \texttt{(list 3 2 1)}.

\paragraph{Answer 10:} Please put your answer in
\texttt{ps\PSnum-lastname-firstname.clj}.

\noindent\hrulefill %--------------------------------

\paragraph{Problem 11:}
Define a function called \required{\texttt{count-to-n}}, that takes a positive
integer \texttt{n}, and returns a list of the integers from \texttt{1} to
\texttt{n}. For example, given input \texttt{3}, it will return the value of
\texttt{(list 1 2 3)}.

\paragraph{Hint:}  Use the procedures \texttt{reverse-list}  and
\texttt{count-to-1} that you wrote in the previous problems.

\paragraph{Answer 11:} Please put your answer in
\texttt{ps\PSnum-lastname-firstname.clj}.

\noindent\hrulefill %--------------------------------

\paragraph{Problem 12:}
Define a function called \required{\texttt{get-max}}, that takes a list of
numbers, and returns the maximum value.  So, \texttt{(get-max '(2 3 3))} should
return \texttt{3}.

Don't use the built-in \texttt{max} function.

\paragraph{Answer 12:} Please put your answer in
\texttt{ps\PSnum-lastname-firstname.clj}.

\noindent\hrulefill %--------------------------------

\paragraph{Problem 13:}
Define a function called \required{\texttt{greater-than-five?}}, that takes a
list of numbers, and returns a list of equal length to the input list, but where
each number is replaced with \texttt{true} if the number is greater than
\texttt{5}, and \texttt{false} otherwise. For example, given input \texttt{(list
5 4 7)}, it will return the list \texttt{'(false false true)}.

\paragraph{Hint:} Use the built in function \texttt{map}.

\paragraph{Answer 13:} Please put your answer in
\texttt{ps\PSnum-lastname-firstname.clj}.

\noindent\hrulefill %--------------------------------

\paragraph{Problem 14:}
Define a function called \required{\texttt{concat-three}}, that takes three
sequences (represented as lists), \texttt{x}, \texttt{y}, and \texttt{z}, and
returns the concatenation of the three sequences. For example, given the
arguments \texttt{(list 'a 'b)}, \texttt{(list 'b 'c)}, and \texttt{(list 'd
'e)}, the procedure should return the value of \texttt{(list 'a 'b 'b 'c 'd
'e)}.

Don't use the built-in \texttt{concat} function.

\paragraph{Answer 14:} Please put your answer in
\texttt{ps\PSnum-lastname-firstname.clj}.

\noindent\hrulefill %--------------------------------

\paragraph{Problem 15:}
Define a function called \required{\texttt{sequence-to-power}}, that takes a
sequence (represented as a list) \texttt{x}, and a nonnegative integer
\texttt{n}, and returns the sequence \texttt{x}$^n$. For example, given the
sequence \texttt{(list 'a 'b)} as the first argument and  the number \texttt{3}
as the second, the procedure should return the value of \texttt{(list 'a 'b 'a
'b 'a 'b)}.

You may use the built-in \texttt{concat} function for this problem.

\paragraph{Answer 15:} Please put your answer in
\texttt{ps\PSnum-lastname-firstname.clj}.

\noindent\hrulefill %--------------------------------

\paragraph{Problem 16:}
Define $L$ as a language containing a single sequence, $L=\{a\}$.

In Clojure, we can represent the sequence $a$ as the list \texttt{'(a)}.

Define a function called \required{\texttt{in-L-star?}} that takes a sequence
(represented as a list), and returns true if and only if the sequence is a
member of the language $L^*$. For example, given a sequence such as \texttt{'(a
b)}, the procedure should return \texttt{false}, because $ab$ is not a member of
$L^*$.

\paragraph{Answer 16:} Please put your answer in
\texttt{ps\PSnum-lastname-firstname.clj}.

\noindent\hrulefill %--------------------------------

\paragraph{Problem 17:}
Let $A$ and $B$ be two distinct formal languages. We'll use $(A\cdot B)$ to
denote the concatenation of $A$ and $B$, in that order. Find an example of
languages $A$ and $B$ such that $(A\cdot B)=(B\cdot A)$.

\paragraph{Answer 17:} Please put your answer in the box below.

\begin{AnswerBox}%% Do not delete %%%%%%%

  %%%%%%%%%%%%%%%%%%%%%%%
  %% YOUR ANSWER HERE. %%
  %%%%%%%%%%%%%%%%%%%%%%%

\end{AnswerBox}%% Do not delete %%%%%%%

\noindent\hrulefill %--------------------------------

\paragraph{Problem 18:}
Let $A$ and $B$ be languages. Find an example of languages $A$ and $B$ such that
$(A\cdot B)$ does not equal $(B\cdot A)$

\paragraph{Answer 18:} Please put your answer in the box below.

\begin{AnswerBox}%% Do not delete %%%%%%%

  %%%%%%%%%%%%%%%%%%%%%%%
  %% YOUR ANSWER HERE. %%
  %%%%%%%%%%%%%%%%%%%%%%%

\end{AnswerBox}%% Do not delete %%%%%%%

\noindent\hrulefill %--------------------------------

\paragraph{Problem 19:}
Find an example of a language $L$ such that $L=L^2$, i.e. $L=(L\cdot L)$.

\paragraph{Answer 19:} Please put your answer in the box below.

\begin{AnswerBox}%% Do not delete %%%%%%%

  %%%%%%%%%%%%%%%%%%%%%%%
  %% YOUR ANSWER HERE. %%
  %%%%%%%%%%%%%%%%%%%%%%%

\end{AnswerBox}%% Do not delete %%%%%%%

\noindent\hrulefill %--------------------------------

\paragraph{Problem 20:}
Argue that the intersection of any two languages $L$ and $L'$ is always
contained in $L$.

\paragraph{Answer 20:} Please put your answer in the box below.

\begin{AnswerBox}%% Do not delete %%%%%%%

  %%%%%%%%%%%%%%%%%%%%%%%
  %% YOUR ANSWER HERE. %%
  %%%%%%%%%%%%%%%%%%%%%%%

\end{AnswerBox}%% Do not delete %%%%%%%

\noindent\hrulefill %--------------------------------

\paragraph{Problem 21:}
Let $L_1$, $L_2$, $L_3$, and $L_4$ be languages. Argue that the union of
Cartesian products $(L_1 \times L_3) \cup (L_2 \times L_4)$ is always contained
in the Cartesian product of unions $(L_1 \cup L_2) \times (L_3 \cup L_4)$.

\paragraph{Answer 21:} Please put your answer in the box below.

\begin{AnswerBox}%% Do not delete %%%%%%%

  %%%%%%%%%%%%%%%%%%%%%%%
  %% YOUR ANSWER HERE. %%
  %%%%%%%%%%%%%%%%%%%%%%%

\end{AnswerBox}%% Do not delete %%%%%%%

\noindent\hrulefill %--------------------------------

\paragraph{Problem 22:}
Let $L$ and $L'$ be finite languages. Show that the number of elements in the
Cartesian product $L \times L'$ is always equal to the number of elements in $L'
\times L$.

\paragraph{Answer 22:} Please put your answer in the box below.

\begin{AnswerBox}%% Do not delete %%%%%%%

  %%%%%%%%%%%%%%%%%%%%%%%
  %% YOUR ANSWER HERE. %%
  %%%%%%%%%%%%%%%%%%%%%%%

\end{AnswerBox}%% Do not delete %%%%%%%

\noindent\hrulefill %--------------------------------

\paragraph{Problem 23:}

Suppose $L$ is a language, and that concatenation of $L$ with itself is equal to
itself: $(L\cdot L) = L$. Show that $L$ is either the empty set,  the set
$\{\epsilon\}$, or an infinite language.

\paragraph{Answer 23:} Please put your answer in the box below.

\begin{AnswerBox}%% Do not delete %%%%%%%

  %%%%%%%%%%%%%%%%%%%%%%%
  %% YOUR ANSWER HERE. %%
  %%%%%%%%%%%%%%%%%%%%%%%

\end{AnswerBox}%% Do not delete %%%%%%%


%--------------------------------
\newpage
\section*{LONG FORM READING QUESTION:} 
\textbf{(This section is optional for students in LING/COMP 445, but must be completed if taking LING 645.)}

You must answer this question on your own.

Andreas et al. (2022) lay out a framework for how to think about data collection and ML model design when trying to study language representation. Though they use sperm whale communication as their driving example, much of the points they make are applicable to any communication system, including humans. What is the difference between supervised learning and self-supervised learning in ML models? What are some of the benefits and downfalls of using self-supervision to train a model of language representation? (give at least 2 benefits and 2 downfalls - your answer should be 1/2 a page to a page in length.)


\paragraph{Answer:} Please put your answer in the box below.

\begin{AnswerBox}%% Do not delete %%%%%%%

  %%%%%%%%%%%%%%%%%%%%%%%
  %% YOUR ANSWER HERE. %%
  %%%%%%%%%%%%%%%%%%%%%%%

\end{AnswerBox}%% Do not delete %%%%%%%

\end{document}
