\documentclass[10pt]{article}
 
\usepackage[margin=1in]{geometry} 
\usepackage{amsmath,amsthm,amssymb, graphicx, multicol, array, parskip}
\usepackage[T1]{fontenc}
\usepackage{textcomp}
\usepackage{url}
% use straight quotes in texttt environment, make 0 different from O
\usepackage[zerostyle=b,straightquotes,scaled=.93]{newtxtt}

\usepackage{listings}
\lstset{
 basicstyle=\ttfamily,
 columns=fullflexible,
 upquote,
 keepspaces,
 literate={*}{{\char42}}1
 {-}{{\char45}}1
}
\usepackage[short labels]{enumitem}

\usepackage{soul}  % for strike through (\st{})
\usepackage[dvipsnames,usenames,table]{xcolor} % for colors
\usepackage[htt]{hyphenat} % to break lines in texttt

\usepackage[framemethod=TikZ]{mdframed}
\usepackage{mdframed}
\global\mdfdefinestyle{simple}{linewidth=1pt,skipabove=.5em}
\newenvironment{AnswerBox}{\begin{mdframed}[style=simple]}{\end{mdframed}}
\newcommand\defeq{\mathrel{\overset{\makebox[0pt]{\mbox{\normalfont\tiny\sffamily def}}}{=}}}
\usepackage[colorlinks]{hyperref} % \href{http://link.com}{link text}
\hypersetup{linkcolor=NavyBlue,citecolor=NavyBlue,filecolor=NavyBlue,urlcolor=NavyBlue}
\usepackage{dsfont}
\newcommand{\required}[1]{{\color{blue}{#1}}}
\newcommand{\email}[1]{\href{mailto:#1}{\texttt{#1}}}
\newcommand{\PSnum}{4}

\author{
  \textbf{Name}:       %Put your name here.
, \textbf{McGill ID}:  %Put your McGill ID here.
\\ \textit{Collaborators}:  %Put the names of up to 2 people if you completed the assignment as a group.
}

\begin{document}

\title{LING/COMP 445, LING 645\\Problem Set \PSnum}
\date{Due before 4:35 PM on Wednesday, November 8, 2023}
\maketitle
Please enter your name and McGill ID above.
There are several types of questions below. 
\begin{itemize}
\item
For questions involving answers in English or mathematics or a
combination of the two, put your answers to the question in an
answer box like in the example below. You can find more
information about \LaTeX{} here \url{https://www.latex-project.org/}.

\item For programming questions,
please put your answers into a file called
\texttt{ps\PSnum-lastname-firstname.clj}. Be careful to follow the instructions
exactly and be sure that all of your function definitions use the
precise names, number of inputs and input types, and output types as
requested in each question.

For the code portion of the assignment, \textbf{it is crucial to submit a
standalone file that runs}. Before you submit \texttt{ps\PSnum-lastname-firstname.clj}, 
make sure that your code executes correctly without any errors 
when run at the command line by typing 
\texttt{clojure ps\PSnum-lastname-firtname.clj} at a terminal
prompt. We cannot grade any code that does not run correctly as a
standalone file, and if the preceding command produces an error,
the code portion of the assignment will receive a $0$.

To do the computational problems, we recommend that you install
Clojure on your local machine and write and debug the answers to each
problem in a local copy of \texttt{ps\PSnum-lastname-firstname.clj}. You can
find information about installing and using Clojure here
\url{https://clojure.org/}.
\end{itemize}
Once you have entered your answers, please compile your copy of this
\LaTeX{} into a PDF and submit 
\begin{enumerate}[(i),noitemsep]
\item
the compiled PDF renamed to
\texttt{ps\PSnum-lastname-firstname.pdf} 
\item
the raw \LaTeX{} file renamed to
\texttt{ps\PSnum-lastname-firstname.tex} and 
\item
your \texttt{ps\PSnum-lastname-firstname.clj}
\end{enumerate}
to the Problem Set \PSnum{} folder under `Assignments' on MyCourses.

\hrulefill %--------------------------------

\paragraph{Example Problem:}
This is an example question using some fake math like this
$L=\sum_0^{\infty} \mathcal{G} \delta_x$.

\paragraph{Example Answer:} Put your answer in the box provided, like this:
\begin{AnswerBox}
Example answer is $L=\sum_0^{\infty} \mathcal{G} \delta_x$.
\end{AnswerBox}



\hrulefill%--------------------------------

\pagebreak%

\paragraph{Problem 1:}
In this problem set, we are going to be considering a variant of the
hierarchical bag-of-words model that we looked at in class. In class,
we used a Dirichlet distribution to define a prior distribution over
$\theta$, the parameter vector of the bag of words model. The
Dirichlet distribution is a continuous distribution on the
simplex---it assigns probability density to all the uncountably many
points on the simplex.

For this problem set, we will be looking at a considerably simpler
prior distribution over the parameters $\theta$. Our distribution will
be \emph{discrete}, and in particular will only assign positive
probability to a finite number of values of $\theta$. Further, for the
purposes of this problem set where we will only be scoring strings
rather than generating them, we will ignore the probability of the
\texttt{'stop} symbol.

The probability distribution is defined in the code below:

\begin{lstlisting}
(def vocabulary '(call me ishmael))

(def theta1 (list (/ 1 2 ) (/ 1 4 ) (/ 1 4 )))
(def theta2 (list (/ 1 4 ) (/ 1 2 ) (/ 1 4 )))

(def thetas (list theta1 theta2))

(def theta-prior (list (/ 1 2) (/ 1 2)))
\end{lstlisting}

Our vocabulary in this case consists of three words. Each value of
$\theta$ therefore defines a bag of words distribution over sentences
containing these three words. The first value of $\theta$
(\texttt{theta1}) assigns $\frac{1}{2}$ probability to the word
\texttt{'call}, $\frac{1}{4}$ to \texttt{'me}, and $\frac{1}{4}$ to
\texttt{'ishmael}. The second value of $\theta$ (\texttt{theta2})
assigns $\frac{1}{2}$ probability to \texttt{'me}, and $\frac{1}{4}$
to each of the other two words. The two values of $\theta$ each have
prior probability of $\frac{1}{2}$. \textbf{Assume throughout the
  problem set that the vocabulary and possible values of $\theta$ are
  fixed to these values above.}

In addition to the code above we will be using some helper functions defined in class:

\begin{lstlisting}
(defn score-categorical [outcome outcomes params]
  (if (empty? params)
    (throw "no matching outcome")
    (if (= outcome (first outcomes))
      (first params)
      (score-categorical outcome (rest outcomes) (rest params)))))

(defn list-foldr [f base lst]
  (if (empty? lst)
    base
    (f (first lst)
       (list-foldr f base (rest lst)))))

(defn log2 [n]
  (/ (Math/log n) (Math/log 2)))

(defn score-BOW-sentence [sen probabilities]
  (list-foldr
   (fn [word rest-score]
     (+ (log2 (score-categorical word vocabulary probabilities))
        rest-score))
   0
   sen))

(defn score-corpus [corpus probabilities]
  (list-foldr
   (fn [sen rst]
     (+ (score-BOW-sentence sen probabilities) rst))
   0
   corpus))

(defn logsumexp [log-vals]
  (let [mx (apply max log-vals)]
    (+ mx
       (log2
        (apply +
               (map (fn [z] (Math/pow 2 z))
                    (map (fn [x] (- x mx)) log-vals)))))))
\end{lstlisting}

Recall that the function \texttt{score-corpus} is used to compute the
log probability of a corpus given a particular value of the parameters
$\theta$. Also recall (from the Discrete Random Variables module) the
purpose of the function \texttt{logsumexp}, which is used to compute
the sum of log probabilities; you should return to the lecture notes
if you don't remember what this function is doing. (Note that the
version of \texttt{logsumexp} here differs slightly from the lecture
notes, as it does not use the \texttt{\&} notation, so it takes one
argument, \texttt{log-vals}, a list of log probabilities.)

Our initial corpus will consist of two sentences:

\begin{lstlisting}
(def my-corpus '((call me)
                 (call ishmael)))
\end{lstlisting}

Write a function \required{\texttt{theta-corpus-joint}}, which takes
three arguments: \texttt{theta}, \texttt{corpus}, and
\texttt{theta-probs}. The argument \texttt{theta} is a value of the
model parameters $\theta$, and the argument \texttt{corpus} is a list
of sentences. The argument \texttt{theta-probs} is a prior probability
distribution over the values of $\theta$. The function should return
the \textbf{log} of the joint probability
$\Pr(C=\texttt{corpus},\theta=\texttt{theta})$.

Use the chain-rule identity discussed in class: $\Pr(C,\theta) =
\Pr(C|\theta)\Pr(\theta)$. Assume that the prior distribution
$\Pr(\theta)$ is defined by the probabilities in \texttt{theta-probs},
which is a list containing the prior probability of each value of
$\theta$ (that is, a list with two entries, one for the probability of
\texttt{theta1} and one for \texttt{theta2}).

After defining this function, you can call \texttt{(theta-corpus-joint
theta1 my-corpus theta-prior)}. This will compute log joint
probability of the model parameters \texttt{theta1} and the corpus
\texttt{my-corpus}.

\paragraph{Answer 1:} Please put your answer in \texttt{ps\PSnum-lastname-firstname.clj}.

\hrulefill %--------------------------------

\paragraph{Problem 2:}

Write a function \required{\texttt{compute-marginal}}, which takes two
arguments: \texttt{corpus} and \texttt{theta-probs}. The argument
\texttt{corpus} is a list of sentences, and the argument
\texttt{theta-probs} is a prior probability distribution on values of
$\theta$.  The function should return the \textbf{log} of the marginal
likelihood of the corpus, when the prior distribution on $\theta$ is
given by theta-probs. That is, the function should return
$\log[\sum_{\theta \in \Theta} \Pr(\mathbf{C}=\texttt{corpus},
\Theta=\theta)]$.

 Hint: Use the \texttt{logsumexp} function defined above.

 After defining \texttt{compute-marginal}, you can call
\texttt{(compute-marginal my-corpus theta-prior)}. This will compute
the marginal likelihood of \texttt{my-corpus} (which was defined
above), given the prior distribution \texttt{theta-prior}.

\paragraph{Answer 2:} Please put the answer in
\texttt{ps\PSnum-lastname-firstname.clj}.

\hrulefill %--------------------------------

\paragraph{Problem 3:}

Write a function \required{\texttt{compute-conditional-prob}}, which
takes three arguments: \texttt{theta}, \texttt{corpus}, and
\texttt{theta-probs}. The arguments have the same interpretation as in
Problems 1 and 2. The function should return the \textbf{log} of the
conditional probability of the parameter value \texttt{theta}, given
the corpus. Remember that the conditional probability is defined by
the equation:

\begin{equation}
  \Pr(\Theta=\theta|\mathbf{C}=\texttt{corpus})
  = \frac{\Pr(\mathbf{C}=\texttt{corpus},\Theta=\theta)}
         {\sum_{\theta\in\Theta}\Pr(\mathbf{C}=\texttt{corpus},\Theta=\theta)}
\end{equation}

[Note: don't forget that your \texttt{compute-conditional-prob}
should return a \textbf{log} probability.]

\paragraph{Answer 3:} Please put your answer in \texttt{ps\PSnum-lastname-firstname.clj}.

\hrulefill %--------------------------------

\paragraph{Problem 4:}
Write a function \required{\texttt{compute-conditional-dist}}, which
takes two arguments: \texttt{corpus} and \texttt{theta-probs}. For
every value of $\theta$ in \texttt{thetas} (i.e., \texttt{theta1} and
\texttt{theta2}), it should return the log conditional probability of
$\theta$ given the corpus. That is, it should return a two-element
list of log conditional probabilities, one for each of the two values
of $\theta$.

\paragraph{Answer 4:} Please put your answer in
\texttt{ps\PSnum-lastname-firstname.clj}.

\hrulefill %--------------------------------

\paragraph{Problem 5:}
Call \texttt{(compute-conditional-dist my-corpus theta-prior)}. What
do you notice about the conditional distribution over values of
$\theta$?  You may want to exponentiate the values you get back, so
that you can see the regular probabilities, rather than the log
probabilities. Explain why the conditional distribution looks the way
it does, with reference to the properties of \texttt{my-corpus}.  In
particular, if one value of $\theta$ has higher conditional probability
than the other, say why.

\paragraph{Answer 5:} Please put your answer in the box below.

\begin{AnswerBox}%% Do not delete %%%%%%%

    %%%%%%%%%%%%%%%%%%%%%%%
    %% YOUR ANSWER HERE. %%
    %%%%%%%%%%%%%%%%%%%%%%%
    
\end{AnswerBox}%% Do not delete %%%%%%%

\hrulefill %--------------------------------

\paragraph{Problem 6:}
When you call \texttt{compute-conditional-dist}, you get back a log
probability distribution over values of $\theta$ (the conditional
distribution over $\theta$ given an observed corpus). This is a
probability distribution just like any other. In particular, it can be
used as the prior distribution over values of $\theta$ in a
hierarchical bag of words model. Given this new hierarchical BOW
model, we can do all of the things that we normally do with such a
model. In particular, we can compute the marginal likelihood of a
corpus under this model. This marginal likelihood is called a
\emph{posterior predictive distribution}.

 Below we have defined the skeleton of a function
\required{\texttt{compute-posterior-predictive}}, which you must
complete. It takes three arguments: \texttt{observed-corpus},
\texttt{new-corpus}, and \texttt{theta-probs}. The argument
\texttt{observed-corpus} is a corpus which we have observed, and are
using to compute a conditional distribution over values of $\theta$.
Given this conditional distribution over $\theta$, we will then
compute the marginal likelihood of the corpus \texttt{new-corpus}. The
function \texttt{compute-posterior-predictive} should return the
marginal log likelihood of the new corpus given the conditional
distribution on $\theta$.

\begin{lstlisting}
(defn compute-posterior-predictive [observed-corpus new-corpus theta-probs]
  (let [conditional-dist ...
    (compute-marginal ...
\end{lstlisting}

 Once you have implemented
\texttt{compute-posterior-predictive}, call
\texttt{(compute-posterior-predictive my-corpus my-corpus
theta-prior)}. What does this quantity represent? How does its value
compare to the marginal likelihood that you computed in Problem 2? Why
is this to be expected?

\paragraph{Answer 6:} Please put your code in
\texttt{ps\PSnum-lastname-firstname.clj} and write the text part of
the answer in the box below.

\begin{AnswerBox}%% Do not delete %%%%%%%

    %%%%%%%%%%%%%%%%%%%%%%%
    %% YOUR ANSWER HERE. %%
    %%%%%%%%%%%%%%%%%%%%%%%
    
\end{AnswerBox}%% Do not delete %%%%%%%

\hrulefill %--------------------------------

\paragraph{Problem 7:}
In the previous problems, we have written code that will compute
marginal and conditional distributions \emph{exactly}, by enumerating
over all possible values of $\theta$. In the next problems, we will
develop an alternate approach to computing these distributions.
Instead of computing these distributions exactly, we will approximate
them using random sampling.  \\

 The following functions were defined in class, and will be
useful for us going forward:

\begin{lstlisting}
(defn normalize [params]
  (let [sum (apply + params)]
    (map (fn [x] (/ x sum)) params)))

(defn flip [weight]
  (if (< (rand 1) weight)
    true
    false))

(defn sample-categorical [outcomes params]
  (if (flip (first params))
    (first outcomes)
    (sample-categorical (rest outcomes) 
                        (normalize (rest params)))))

(defn sample-BOW-sentence [len probabilities]
  (if (= len 0)
    '()
    (cons (sample-categorical vocabulary probabilities)
          (sample-BOW-sentence (- len 1) probabilities))))
\end{lstlisting}

 Recall that the function \texttt{sample-BOW-sentence}
samples a sentence from the bag of words model of length \texttt{len},
given the parameter vector \texttt{probabilities}.  \\

 Define a function \required{\texttt{sample-BOW-corpus}},
which takes three arguments: \texttt{theta}, \texttt{sent-len}, and
\texttt{corpus-len}. The argument \texttt{theta} is a value of the
model parameters $\theta$. The arguments \texttt{sent-len} and
\texttt{corpus-len} are positive integers. The function should return
a sample corpus from the bag of words model, given the model
parameters \texttt{theta}. Each sentence should be of length
\texttt{sent-len} and number of sentences in the corpus should be
equal to \texttt{corpus-len}. For example, if \texttt{sent-len} equals
3 and \texttt{corpus-len} equals 2, then this function should return a
list of 2 sentences, each consisting of 3 words.  \\

 Hint: Use \texttt{sample-BOW-sentence}. You may also want to
use the built-in function \texttt{repeatedly}.

\paragraph{Answer 7:} Please put your answer in
\texttt{ps\PSnum-lastname-firstname.clj}.

\hrulefill %--------------------------------

\paragraph{Problem 8:}

Below we have defined the skeleton of the function
\required{\texttt{sample-theta-corpus}} which you must complete.  This
function takes three arguments: \texttt{sent-len} \texttt{corpus-len}
and \texttt{theta-probs}. It returns a list with two elements: a value
of $\theta$ sampled from the distribution defined by
\texttt{theta-probs}; and a corpus sampled from the bag of words model
given the sampled $\theta$. (The number of sentences in the corpus
should equal \texttt{corpus-len}, and each sentence should have
\texttt{sent-len} words in it.) \\

 We will call the return value of this function a \texttt{theta-corpus} pair.

\begin{lstlisting}
(defn sample-theta-corpus [sent-len corpus-len theta-probs]
  (let [theta ...
    (list theta ...
\end{lstlisting}

\paragraph{Answer 8:} Please put your answer in
\texttt{ps\PSnum-lastname-firstname.clj}.

\hrulefill %--------------------------------

\paragraph{Problem 9:}

Below we have defined some useful functions for us. The function
\texttt{get-theta} takes a \texttt{theta-corpus} pair, and returns the
value of \texttt{theta} in it. The function \texttt{get-corpus} takes
a \texttt{theta-corpus} pair, and returns its corpus value. The
function \texttt{sample-thetas-corpora} samples multiple theta-corpus
pairs, and returns a list of them. In particular, the number of
samples it returns equals sample-size.  The function \texttt{get-count}
counts the number of times an outcome appears in a list, and will be
useful in this problem as well as Problem 11.

\begin{lstlisting}
(defn get-theta [theta-corpus]
  (first theta-corpus))

(defn get-corpus [theta-corpus]
  (first (rest theta-corpus)))
  
(defn sample-thetas-corpora [sample-size sent-len corpus-len theta-probs]
  (repeatedly sample-size (fn [] (sample-theta-corpus sent-len corpus-len theta-probs))))

(defn get-count [outcome lst]
  (let [filtered-lst 
        (filter (fn [x] (= outcome x)) lst)]
  (count filtered-lst)))
\end{lstlisting}

 We are now going to estimate the marginal likelihood of a
corpus by using random sampling. Here is the general approach that we
are going to use. We are going to sample some number (for example
1000) of \texttt{theta-corpus} pairs. These are 1000 samples from the
joint distribution defined by the hierarchical bag of words model. We
are then going to throw away the values of \texttt{theta} that we
sampled; this will leave us with 1000 corpora sampled from our model.

We are going to use these 1000 sampled corpora to estimate
the probability of a specific target corpus. The process here is
simple. We just count the number of times that our target corpus
appears in the 1000 sampled corpora. The ratio of the occurrences of
the target corpus to the number of total corpora gives us an estimate
of the target's probability.

More formally, let us suppose that we are given a target
corpus $\mathbf{t}$. We will define the indicator function
$\mathds{1}_{\mathbf{t}}$ by:

\begin{equation}
\mathds{1}_{\mathbf{t}}(c) =
\begin{cases}
    1, & \text{if } t = c\\
    0, & \text{otherwise}
\end{cases}
\end{equation}

 We will sample $n$ corpora $c_1,\ldots,c_n$ from the
hierarchical bag of words model. We will estimate the marginal
likelihood of the target corpus $\mathbf{t}$ by the following formula:

\begin{equation}
\label{eq:montecarlo-marginal}
\sum_{\theta \in \Theta} \Pr(\mathbf{C}=\mathbf{t}, \Theta=\theta)  \approx \frac{1}{n} \sum_{i}^{n}\mathds{1}_{\mathbf{t}}(c_i) 
\end{equation}

 Define a procedure
\required{\texttt{estimate-corpus-marginal}}, which takes five
arguments: \texttt{corpus}, \texttt{sample-size}, \texttt{sent-len},
\texttt{corpus-len}, and \texttt{theta-probs}. The argument
\texttt{corpus} is the target corpus whose marginal likelihood we want
to estimate. \texttt{sample-size} is the number of corpora that we are
going to sample from the hierarchical model (its value was 1000 in the
discussion above). The arguments \texttt{corpus-len} and
\texttt{sent-len} characterize the number of sentences in the corpus
and the number of words in each sentence, respectively. The argument
\texttt{theta-probs} is the prior probability distribution over
$\theta$ for our hierarchical model.

The procedure should return an estimate of the marginal
\textbf{(not log)} likelihood of the target corpus, using the formula
defined in Equation \ref{eq:montecarlo-marginal}.  \\

 Hint: Use \texttt{sample-thetas-corpor}a to get a list of
samples of \texttt{theta-corpus pairs}, and then use
\texttt{get-corpus} to extract the corpus values from these pairs (and
ignore the \texttt{theta} values).

\paragraph{Answer 9:} Please put your answer in
\texttt{ps\PSnum-lastname-firstname.clj}.

\hrulefill %--------------------------------

\paragraph{Problem 10:}

 Call \texttt{(estimate-corpus-marginal my-corpus 50 2 2
  theta-prior)} a number of times. What do you notice? Now call
\texttt{(estimate-corpus-marginal my-corpus 10000 2 2 theta-prior)} a
number of times. How do these results compare to the previous ones?
How do these results compare to the exact marginal
likelihood that you computed in Problem 2?

\paragraph{Answer 10:} Please put your answer in the box below.

\begin{AnswerBox}%% Do not delete %%%%%%%

    %%%%%%%%%%%%%%%%%%%%%%%
    %% YOUR ANSWER HERE. %%
    %%%%%%%%%%%%%%%%%%%%%%%
    
\end{AnswerBox}%% Do not delete %%%%%%%
 

\hrulefill %--------------------------------

\paragraph{Problem 11:}
 In Problem 9, we introduced a way of approximating the
marginal likelihood of a corpus by using random sampling. We can
similarly approximate a conditional probability distribution by using
random sampling.  \\

 Suppose that we have observed a corpus $\mathbf{c}$, and we
want to compute the conditional probability of a particular $\theta$.
We can approximate this conditional probability as follows. We first
sample $n$ theta-corpus pairs. We then remove all of the pairs in
which the corpus does not match our observed corpus $\mathbf{c}$. We
finally count the number of times that $\theta$ occurs in the
remaining theta-corpus pairs, and divide by the total number of
remaining pairs. This process is an example of \textit{rejection sampling}.
\\

 Define a function \required{\texttt{rejection-sampler}}
which has the following form:
 
\begin{lstlisting}
(defn rejection-sampler
  [theta observed-corpus sample-size sent-len corpus-len theta-probs]
  ...
)
\end{lstlisting}

 This function should use the rejection sampling method (as
described above) to estimate the conditional probability of
\texttt{theta}, given that we have observed the corpus
\texttt{observed-corpus}. The function must estimate this conditional
probability by taking \texttt{sample-size} samples (you may assume
this argument is a positive integer) from the joint distribution on
\texttt{theta-corpus} pairs. The procedure should filter out any
\texttt{theta-corpus} pairs in which the corpus does not equal the
observed corpus. If there are no remaining pairs after filtering, then 
the function should return \texttt{nil}. Otherwise,
it should then count the number of times that \texttt{theta} occurs in the remaining pairs,
and divide by the total number of those pairs.

Hint: Use \texttt{get-count} to count the number of occurrences
of \texttt{theta}.

\paragraph{Answer 11:} Please put your answer to the coding problem in
\texttt{ps\PSnum-lastname-firstname.clj}.

\hrulefill %--------------------------------

\paragraph{Problem 12:}

Call \texttt{(rejection-sampler theta1 my-corpus 100 2 2 theta-prior)}
a number of times. What do you notice? Try with larger sample sizes
(such as $200, 500, 1000\ldots$). How large does
\texttt{sample-size} need to be until you get a stable estimate of the
conditional probability of \texttt{theta1}? Why does it take so many
samples to get a stable estimate?

\paragraph{Answer 12:} Please answer the questions in the box below.

\begin{AnswerBox}%% Do not delete %%%%%%%

    %%%%%%%%%%%%%%%%%%%%%%%
    %% YOUR ANSWER HERE. %%
    %%%%%%%%%%%%%%%%%%%%%%%
    
\end{AnswerBox}%% Do not delete %%%%%%%


%--------------------------------
\newpage
\section*{LONG FORM READING QUESTION:} 
\textbf{(This section is optional for students in LING/COMP 445, but must be completed if taking LING 645.)}

You must answer this question on your own.

Based on Warstadt \& Bowman 2022 and our discussion in class, choose two differences between human learners' and language models' learning environment that you think are the most critical for language learning. Discuss both (1) why you think they are the most important and (2) how models may be modified to account for these differences -- this second point can be hypothetical if no models have addressed them yet. 

\paragraph{Answer:} Please put your answer in the box below.

\begin{AnswerBox}%% Do not delete %%%%%%%

  %%%%%%%%%%%%%%%%%%%%%%%
  %% YOUR ANSWER HERE. %%
  %%%%%%%%%%%%%%%%%%%%%%%

\end{AnswerBox}%% Do not delete %%%%%%%


\end{document}

